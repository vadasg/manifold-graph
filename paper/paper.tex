\documentclass[12pt]{article}

\usepackage[margin=1in]{geometry}  % set the margins to 1in on all sides
\usepackage{graphicx}              % to include figures
\usepackage{amsmath}               % great math stuff
\usepackage{amsfonts}              % for blackboard bold, etc
\usepackage{amsthm}                % better theorem environments
\usepackage{url}

% various theorems, numbered by section

\newtheorem{thm}{Theorem}[section]
\newtheorem{lem}[thm]{Lemma}
\newtheorem{prop}[thm]{Proposition}
\newtheorem{cor}[thm]{Corollary}
\newtheorem{conj}[thm]{Conjecture}

\DeclareMathOperator{\id}{id}

\newcommand{\bd}[1]{\mathbf{#1}}  % for bolding symbols
\newcommand{\RR}{\mathbb{R}}      % for Real numbers
\newcommand{\ZZ}{\mathbb{Z}}      % for Integers
\newcommand{\col}[1]{\left[\begin{matrix} #1 \end{matrix} \right]}
\newcommand{\comb}[2]{\binom{#1^2 + #2^2}{#1+#2}}
	

\begin{document}


\nocite{*}

\title{A Generalized Sphere Theorem for Positively Curved Combinatorial 3-Manifolds}

\author{Aaron Trout \\ Department of Mathematics \\
Chatham University \\ Woodland Rd, Pittsburgh PA 15232 USA \and
Vadas Gintautas\\ Department of Physics \\
Chatham University \\ Woodland Rd, Pittsburgh PA 15232 USA}


\maketitle

\begin{abstract}
  This paper presents a discrete version of Grove and Shiohama's generalized sphere theorem \cite{groveshiohama}.
\end{abstract}


\section{Introduction}

Differential geometry is of central importance not only to geometers but also topologists, physicists and, increasingly, many of those interested in applied topics such as finite element analysis REFERENCE and computer graphics REFERENCE. One new offshoot in this area is {\em discrete differential geometry} (DDG) which seeks discrete analogs to the classical theorems and concepts from differential geometry. Since many way computational treatments of differential geometry involve discretizing shapes, this subject has particular relevance for those with applied interests REFERECE. Other recent DDG work can be found in \cite{BMM}, \cite{Crowley}, \cite{EMM}, \cite{forman2},
\cite{GGL}, \cite{LS} and \cite{stone}. 

A particularly important goal in classical differential geometry is to elucidate the relationship between the curvature of a Riemannian (or semi-Riemannian) space and its topology. The classical
results in this area are numerous, beautiful, and have inspired an
enormous amount of subsequent research (REFERENCES). In this paper, we will present a discrete analog of the ``generalized'' sphere theorem of Grove and Shiohama \cite{groveshiohama}.

\begin{thm}[Grove-Shiohama] Let $M$ be a complete, connected, $n$-dimensional Riemannian manifold with section curvature $K \geq \delta > 0$ and $diam(M) > \frac{\pi}{2\sqrt{\delta}}$. Then, $M$ is homeomorphic to a sphere.
\end{thm}

\noindent Note that the diameter bound in this theorem is exacly half that allowed by the Bonnet-Myers theorem:

\begin{thm}[Bonnet-Myers] Let $M$ be a complete, connected, $n$-dimensional Riemannian manifold with section curvature $K \geq \delta > 0$. Then $diam(M) \leq \frac{\pi}{\sqrt{\delta}}$.
\end{thm}

\section{Basic Stuff}
\label{sect:basics}


\bibliographystyle{plain}

\bibliography{paper}

\begin{thebibliography}{99}

    \bibitem{BMM}N. Brady, J. McCammond, and J. Meier, {\em Bounding edge degrees in triangulated 3-manifolds}, Proc. Amer. Math. Soc. 132 (2004), no. 1, 291--298 (electronic).

    \bibitem{cheng}Y. Cheng, {\em Eigenvalue comparison theorems and its [sic] geometric applications}, Math. Z. 143 (1975) 289-297.

    \bibitem{Crowley}K. Crowley, Doctoral Thesis (Rice University), {\em Discrete Morse Theory and the Geometry of Nonpositively Curved Simplicial Complexes}. (2001).

    \bibitem{EMM}M. Elder, J. McCammond, and J. Meier, {\em Combinatorial conditions that imply word-hyperbolicity for 3-manifolds.}, Topology, 42, (2003), 1241-1259.

    \bibitem{forman1}R. Forman, {\em Morse theory for cell complexes}, Adv. in Math. 134 (1998), 90-145.

    \bibitem{forman2}R. Forman, {\em Combinatorial Differential Topology and Geometry}, New perspectives in algebraic combinatorics (Berkeley, CA, 1996--97), 177--206, Math. Sci. Res. Inst. Publ., 38, Cambridge Univ. Press, Cambridge, 1999.

    \bibitem{GGL}A. Gabrielov, I. Gelfand, and M. Losik.  {\em Combinatorial computation of characteristic classes}. I, II, and {\em A local combinatorial formula for the first Pontrjagin class}. Functional Anal. Appl. 9 (1975), no. 2, 103--116; ibid. 9 (1975), no. 3, 186--202 (1976); ibid. 10 (1976), no. 1, 12--15.

    \bibitem{groveshiohama}K. Grove; K, Shiohama, {\em A generalized sphere theorem}, Ann. of Math. 106 (1977), 201-211.

    \bibitem{gromov}M. Gromov, {\em Curvature, diameter, and Betti numbers}, Comm. Math. Helvetici 56 (1981), 179-195.

    \bibitem{LS}F. Luo and R. Stong, {\em Combinatorics of triangulations of 3-manifolds.} Trans. Amer. Math. Soc. 337 (1993), no. 2, 891-906.

    \bibitem{stone}D.A. Stone, {\em Sectional curvature in piecewise linear manifolds}, Bull. Amer. Math. Soc., vol. 79 (1973), pp. 1060-1063.

    \bibitem{toponogov}V.A. Toponogov, {\em Riemannian spaces having their curvature bounded below by a positive number}, Uspekhi Math. Nauk 14 (1959), 87-135 English Translation: Amer. Math. Soc.Transl. Ser. 37 (1964) 287-290.

    \bibitem{DDS} M. Deza, M. Dutour and M. Shtogrin, {\em On simplicial and cubical complexes with short links}, Israel Journal of Mathematics 144 (2004) 109-
124.

    \bibitem{LutzSul} F. H. Lutz and J. M. Sullivan: {\em Simplicial manifolds with small valences}.
In preparation.


\end{thebibliography}


\end{document}
