\documentclass[12pt]{article}

\usepackage[margin=1in]{geometry}  % set the margins to 1in on all sides
\usepackage{graphicx}              % to include figures
\usepackage{amsmath}               % great math stuff
\usepackage{amsfonts}              % for blackboard bold, etc
\usepackage{amsthm}                % better theorem environments
\usepackage{url}

% various theorems, numbered by section

\newtheorem{thm}{Theorem}[section]
\newtheorem{lem}[thm]{Lemma}
\newtheorem{prop}[thm]{Proposition}
\newtheorem{cor}[thm]{Corollary}
\newtheorem{conj}[thm]{Conjecture}
\newtheorem{dfn}[thm]{Definition}

\DeclareMathOperator{\id}{id}

\newcommand{\bd}[1]{\mathbf{#1}}  % for bolding symbols
\newcommand{\RR}{\mathbb{R}}      % for Real numbers
\newcommand{\ZZ}{\mathbb{Z}}      % for Integers
\newcommand{\col}[1]{\left[\begin{matrix} #1 \end{matrix} \right]}
\newcommand{\comb}[2]{\binom{#1^2 + #2^2}{#1+#2}}
	

\begin{document}


\nocite{*}

\title{A Generalized Sphere Theorem for Positively Curved Combinatorial 3-Manifolds}

\author{Aaron Trout \\ Department of Mathematics \\
Chatham University \\ Woodland Rd, Pittsburgh PA 15232 USA \and
Vadas Gintautas\\ Department of Physics \\
Chatham University \\ Woodland Rd, Pittsburgh PA 15232 USA}


\maketitle

\begin{abstract}
  This paper presents a discrete version of Grove and Shiohama's generalized sphere theorem. TO DO: EXPAND ABSTRACT?
\end{abstract}


\section{Introduction}

Differential geometry is of central importance not only to geometers but also topologists, physicists and, increasingly, many of those interested in applied topics such as finite element analysis and computer graphics. One new offshoot in this area is {\em discrete differential geometry} (DDG) which seeks discrete analogues to the classical theorems and concepts from differential geometry. Since many computational treatments of differential geometry involve discretizing shapes, this subject has particular relevance for those with applied interests, see \cite{grinspun2006discrete} for examples. Other recent DDG work with a more pure-mathematics flavor can be found in \cite{BMM,Crowley,EMM,forman2,GGL1,GGL2,GGL3,stone}. 

A particularly important goal in classical differential geometry is to elucidate the relationship between the curvature of a Riemannian (or semi-Riemannian) space and its topology. The classical
results in this area are numerous, beautiful, and have inspired an
enormous amount of subsequent research. In this paper, we will present a discrete analogue of the ``generalized'' sphere theorem of Grove and Shiohama \cite{groveshiohama}.

\begin{thm}[Grove-Shiohama] Let $M$ be a complete, connected, $n$-dimensional Riemannian manifold with section curvature $K \geq \delta > 0$ and diameter greater than $\frac{\pi}{2\sqrt{\delta}}$. Then, $M$ is homeomorphic to a sphere.
\label{thm:grove_shiohama}
\end{thm}

\noindent Note that this bound is sharp, since the real projective space $\mathbb{RP}^n$ admits a metric with uniform sectional curvature $K=1$ and diameter $\pi/2$. We should also note also that the diameter bound in Theorem \ref{thm:grove_shiohama} is exactly half the maximum diameter allowed by the Bonnet-Myers theorem:

\begin{thm}[Bonnet-Myers] Let $M$ be a complete, connected, $n$-dimensional Riemannian manifold with section curvature $K \geq \delta > 0$. Then the diameter of $M$ is at most $\frac{\pi}{\sqrt{\delta}}$.
\label{thm:bonnet_myers}
\end{thm}

\noindent Our results are derived from brute-force checking of a combinatorial 3-manifold census created by Lutz and Sullivan \cite{LS}.

\section{Basic Stuff}
\label{sect:basics}

The discrete version of Theorem \ref{thm:grove_shiohama} given in this paper will apply to positively curved combinatorial 3-manifolds. A \textbf{combinatorial 3-manifold} $M$ is an abstract simplicial complex in which the link of each vertex is a 2-sphere. We call such a space \textbf{positively curved} if at most five tetrahedra are incident along each edge. Why this terminology? If we endow $M$ with the standard piecewise-linear (PL) metric in which all edges have unit-length, this condition is equivalent to requiring an angle deficit along each edge. In classical differential geometry an angle deficit is intimately related to positive curvature.

A very natural discrete definition of distance in an abstract simplicial complex uses \textbf{edge-paths}. That is, paths entirely contained in the 1-skeleton of $M$.

\begin{dfn}The \textbf{edge-distance} between two vertices $v$ and $w$ in an abstract simplicial complex $M$ is the minimum length (as a PL-path in the standard PL-metric) of an edge-path from $v$ to $w$. We denote this quantity $d_1(v,w)$. The \textbf{edge-diameter} of $M$, written as $diam_1(M)$, is the maximum of $d_1(v,w)$ over all pairs of vertices $v$ and $w$ in $M$. 
\end{dfn}

\noindent Note that the length of an edge-path is simply the number of edges it traverses.

A discrete version of the Bonnet-Myers theorem was previously proved in \cite{Trout10}. It applies to positively curved combinatorial 3-manifolds and gives a diameter bound in terms of the edge-diameter.

\begin{thm}[Trout] For any positively curved combinatorial 3-manifold $M$ we have $diam_1(M)\leq 5$. Moreover, this bound is sharp.
\label{thm:discrete_BM}
\end{thm}

\noindent Interestingly, while the statement of Theorem \ref{thm:discrete_BM} uses edge-diameter, its proof relies on expanding the set of paths under consideration to include those which contain not only edges, but also other types of PL-paths between vertices.

\subsection{Hops and Jumps}
The first new type of path is called a {\em hop}.

\begin{dfn}[Hops] Suppose $\tau$ is a 2-simplex in $M$ and $v_1$ and $v_2$ are vertices in $M$ such that $v_1*\tau$ and $v_2*\tau$ are 3-simplices in $M$. The PL-path from $v_1$ through the barycenter of $\tau$ and ending on $v_2$ will be called a \textbf{hop} from $v_1$ to $v_2$. See Figure \ref{fig:hop}. TO DO: FIX THIS FIG NUM
\end{dfn}

\begin{figure}
	\label{fig:hop}
    \begin{center}
        \includegraphics[width=0.6\linewidth]{figures/hops.pdf}
        \caption{A {\em hop} from vertex $v_1$ to vertex $v_2$. TO DO: CROP OUT LEFT IMAGE}
    \end{center}
\end{figure}

\noindent The other new type of path we call a {\em jump.}

\begin{dfn}[Jumps] Suppose there are edges $e_1$ and $e_2$ and vertices $v_1$ and $v_2$ in $M$ so that $e_1*e_2$ is a 3-simplex in $M$ and $v_1*e_1$ and $v_2*e_2$ are 2-simplices in $M$. We call the PL-path from $v_1$ through the barycenters of $e_1$ and $e_2$ and ending on $v_2$ a \textbf{jump} from $v_1$ to $v_2$. See Figure \ref{fig:jump}. TO DO: FIX THIS FIG NUM
\end{dfn}

\begin{figure}
    \label{fig:jump}
    \begin{center}
        \includegraphics[width=0.4\linewidth]{figures/jump.pdf}
        \caption{A {\em jump} from vertex $v_1$ to vertex $v_2$}
    \end{center}
\end{figure}

\noindent Just as for edges in an edge-path, the length of each hop and jump will be its length as a PL-path in the standard PL metric. Some basic Euclidean geometry tells us these lengths are $H = \frac{2}{3}\sqrt{2}$ and $J = \frac{1}{2}\sqrt{2} + \sqrt{3}$ respectively. We let $d(v,w)$ denote the \textbf{distance between vertices $v$ and $w$} obtained by minimizing over all paths containing edges as well as hops and jumps. Similarly, we let $diam(M)$ denote the \textbf{diameter of $M$} defined in terms of the distance function $d$.

Why are these paths relevant to this paper? The classical generalized sphere theorem requires the manifold to have diameter more than half the maximum allowed by the classical Bonnet-Myers theorem. If we were to naively imitate the classical results in the discrete setting, from Theorem \ref{thm:discrete_BM} we would want the bound $diam_1(M)>\frac{5}{2}$. Since only integral values of edge-distance can occur, this would correspond to $diam_1(M)\geq 3$. Unfortunately, there are positively curved combinatorial 3-manifolds with edge-diameter three that are not homeomorphic to the 3-sphere. In particular, there exists positively curved triangulations of $\mathbb{RP}^3$ with edge-diameter three. However, if we instead use the finer-grained measure of diameter, $diam(M)$ this problem disappears and the discrete results again mirror the classical ones.

From the proof of Theorem \ref{thm:discrete_BM} in \cite{Trout10}, we have the following bound on the possible distances which occur in a positively curved 3-manifold.

\begin{lem}[Trout] If $v$ and $w$ are vertices in a positively curved combinatorial 3-manifold $M$ then $d(v,w) \in \{0, 1, H, 2, J, 3, 2H, 4, 2J \}$.
\end{lem}

\noindent Note that we have listed the possible distances in increasing numerical order. This result implies a diameter bound, which is known to be sharp by results in \cite{Trout10}.

\begin{thm} If $M$ is a positively curved combinatorial 3-manifold then $diam(M) \leq 2J$.
\label{thm:discrete_BM_expanded_paths}
\end{thm}

\noindent This is essentially Theorem \ref{thm:discrete_BM} stated in terms of our new expanded set of paths. It is this result which we will consider our ``discrete Bonnet-Myers'' theorem.

\section{Main Results}

We are now finally able to state our main result, which shows that any positively curved 3-manifold with diameter larger than half that allowed by Corollary \ref{thm:discrete_BM_expanded_paths}.

\begin{thm} Any positively curved combinatorial 3-manifold $M$ with $diam(M)>J$ is homeomorphic to the 3-sphere. This bound is sharp, and equal to half the maximum diameter which occurs for any such $M$.
\label{thm:discrete_GS}
\end{thm}

\noindent The proof of this result relies on a complete census of positively curved combinatorial 3-manifolds completed by Lutz and Sullivan. The census contains 4787 manifolds and contains examples of the 3-sphere $S^3$, the real projective space $\mathbb{RP}^3$, the lens spaces $L(3,1)$ and $L(4,1)$ and the cube space $S^3/Q$. The creation of such a census is a highly non-trivial task, for details see \cite{LutzSul}.

Here is a table of the number of triangulations in the census at each topological type and diameter.\vspace{.1in}

\begin{tabular} {| l l | l l | l l | l l | l l |}
\hline
\multicolumn{2}{|c|}{$S^{3}/Q$} &
\multicolumn{2}{|c|}{$L(3,1)$} &
\multicolumn{2}{|c|}{$RP^{3}$} &
\multicolumn{2}{|c|}{$S^{3}$} &
\multicolumn{2}{|c|}{$L(4,1)$} \\
\hline
\hline
$1H$&1    &$2E$&1    &$1H$&10    &$1E$&3    &$1J$&1 \\
  &       &$1J$&1    &$2E$&7     &$1H$&173  &    &  \\
  &       &  &       &$1J$&5     &$2E$&401  &    &  \\
  &       &  &       &    &      &$1J$&2438 &    &  \\
  &       &  &       &    &      &$3E$&1060 &    &  \\
  &       &  &       &    &      &$2H$&582  &    &  \\
  &       &  &       &    &      &$4E$&93   &    &  \\
  &       &  &       &    &      &$2J$&11   &    &  \\
\hline
\end{tabular}\vspace{.1in}

\noindent Here is a table of the number of triangulations at each diameter. \vspace{.1in}

\begin{tabular} {| l l |}
\hline
$1E$ &       3\\
$1H$ &       184\\
$2E$ &       409\\
$1J$ &       2445\\
$3E$ &       1060\\
$2H$ &       582\\
$4E$ &       93\\
$2J$ &       11\\
\hline
\end{tabular}


\begin{figure}
    \begin{center}
    \includegraphics[width=0.6\linewidth]{figures/global_statistics.png}
    \caption{Global statistics}
    \end{center}
    \label{global_statistics}
\end{figure}


\section{Algorithm Used}


\bibliographystyle{plain}
\bibliography{manifolds}
\end{document}
